\documentclass{article}

% Packages
\usepackage{amsmath}
\usepackage{amsfonts}
\usepackage{graphicx}
\usepackage{float}
\usepackage{wasysym}
\usepackage{fancyhdr}
\usepackage{multirow}
\usepackage{blindtext}
\usepackage{csvsimple,longtable,booktabs}
\usepackage{german}
\usepackage[utf8]{inputenc}
\usepackage[T1]{fontenc}
\usepackage{lscape}
\usepackage{comment}
\usepackage{kvoptions}
\usepackage{textcomp}
\usepackage{url}
\usepackage{geometry}
 \geometry{
 a4paper,
 total={170mm,237mm},
 left=20mm,
 top=30mm,
 }

% commands
\newcommand{\sps}{\;\;\;\;\;\;\;}
\newcommand{\vsps}{\vspace{0.3cm}}
\newcommand{\bet}[1]{\left| #1 \right|}
\newcommand{\rect}[1]{\left[ #1 \right]}
\newcommand{\swir}[1]{\{#1\}}


% creates Header 
\pagestyle{fancy}
    \setlength{\headheight}{13.59999pt}
    \fancyhead{}
    \fancyfoot{}
    \fancyhead[RO,R]{Gruppe X}
    \fancyhead[LO,L]{FHNW$|$HLS}
    \fancyhead[CO,C]{XXX}
    \fancyfoot[RO,R]{\thepage}
    \fancyfoot[LO,L]{\rightmark}
    \renewcommand{\footrulewidth}{0.4pt}




\begin{document}
% Hier wird Seite 1 erstellt
\title{Title}

\date{Date}
\maketitle
\begin{figure}
    \centering
    \includegraphics[width=\textwidth]{}
    \label{fig:my_label}
\end{figure}
\pagebreak[4]



% Hier wird seite 2 eingefügt

\noindent{\Large Titel}\\\\
Praktikum Dialyse\\\\\\\\\\
{\Large Autoren}\\\\
Andreas, Bertschi\\
andreas.bertschi@students.fhnw.ch\\
BSc Medizintechnik (4. Semester)\\\\\\
Janin, Herren\\
janin.herren@students.fhnw.ch\\
BSc Medizintechnik (4. Semester)\\\\\\
Florence, Küng\\
florence.kueng@students.fhnw.ch\\
BSc Medizintechnik (4. Semester)\\\\\\


\noindent{\Large Institution}\\\\
Fachhochschule Nordwestschweiz\\
Hochschule für Life Sciences\\
Campus Muttenz
\pagebreak[4]

% creates Tableofcontents and sets tocdeph to 2
\setcounter{tocdepth}{2}
\tableofcontents

% inputs all following sections
\pagebreak[4]
\section{Einleitung}\label{sec:einleitung}
\blindtext[2]
\pagebreak[4]
\section{Materialien und Methoden}\label{sec:materialien_und_methoden}

\pagebreak[4]
\input{Sections/Theo_Grund}
\pagebreak[4]
\input{Sections/Ergebnisse}
\pagebreak[4]
\section{Diskusion}\label{sec:diskusion}
\pagebreak[4]
\input{Sections/Anhang}

\end{document}
